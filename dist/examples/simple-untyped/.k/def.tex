\nonstopmode
\documentclass{article}
\usepackage[poster,style=bubble]{k}
\title{SIMPLE-UNTYPED}
\begin{document}
\begin{kdefinition}
\maketitle
\begin{module}{\moduleName{SIMPLE-UNTYPED-SYNTAX}}
\begin{kblock}[text]
 \section{Syntax}
We start by defining the SIMPLE syntax.  The language constructs discussed
above have the expected syntax and evaluation strategies.  Recall that in \K
we annotate the syntax with appropriate strictness attributes, thus giving
each language construct the desired evaluation strategy. \end{kblock}
\begin{kblock}[text]
 \subsection{Identifiers}
The special identifier for the function ``main'' belongs to all programs.
Each program may use additional identifiers, which need to be declared either
automatically (when one uses an external parser) or manually
(when one writes the program). \end{kblock}
\begin{kblock}[text]
 \subsection{Declarations}
There are two types of declarations: for variables (including arrays) and
for functions. \end{kblock}

\begin{syntaxBlock}{\nonTerminal{\sort{Start}}}\syntax{{\nonTerminal{\sort{Stmts}}}}{}
\end{syntaxBlock}

\begin{syntaxBlock}{\nonTerminal{\sort{Ids}}}\syntax{List\{Id, \mbox{``},\mbox{''}\}}{strict, hybrid}
\end{syntaxBlock}

\begin{syntaxBlock}{\nonTerminal{\sort{Exps}}}\syntax{List\{Exp, \mbox{``},\mbox{''}\}}{}
\end{syntaxBlock}

\begin{syntaxBlock}{\nonTerminal{\sort{Decl}}}\syntax{\terminal{var}{{\nonTerminal{\sort{Exps}}}}\terminal{;}}{}\syntaxCont{\terminal{function}{{\nonTerminal{\sort{Id}}}}({{\nonTerminal{\sort{Ids}}}}){{\nonTerminal{\sort{Stmt}}}}}{}
\end{syntaxBlock}
\begin{kblock}[text]
 \subsection{Expressions}
The expression constructs below are standard.  Increment (\texttt{++}) takes
an expression rather than a variable because it can also increment an array
element.  Arrays can be multidimensional and can hold other arrays, so their
lookup operation takes a list of expressions as argument and applies to an
expression (which can in particular be another array lookup), respectively.
The construct \texttt{sizeof} gives the size of an array in number of elements
of its first dimension.  Note that almost all constructs are strict.
Exceptions are the increment (since its first argument gets updated, so it
cannot be evaluated) and the assignment which is only strict in its second
argument (for the same reason as the increment). \end{kblock}

\begin{syntaxBlock}{\nonTerminal{\sort{Exp}}}\syntax{{\nonTerminal{\sort{Int}}}}{}\syntaxCont{{\nonTerminal{\sort{Bool}}}}{}\syntaxCont{{\nonTerminal{\sort{Id}}}}{}\syntaxCont{{\nonTerminal{\sort{String}}}}{}\syntaxCont{({\nonTerminal{\sort{Exp}}})}{}\syntaxCont{\terminal{++}{{\nonTerminal{\sort{Exp}}}}}{}\syntaxCont{{{\nonTerminal{\sort{Exp}}}}[{{\nonTerminal{\sort{Exps}}}}]}{strict}\syntaxCont{{{\nonTerminal{\sort{Exp}}}}({{\nonTerminal{\sort{Exps}}}})}{strict}\syntaxCont{\terminal{-}{{\nonTerminal{\sort{Exp}}}}}{strict}\syntaxCont{\terminal{sizeOf}({{\nonTerminal{\sort{Exp}}}})}{strict}\syntaxCont{\terminal{read}()}{}\syntaxCont{{{\nonTerminal{\sort{Exp}}}}\terminal{*}{{\nonTerminal{\sort{Exp}}}}}{strict}\syntaxCont{{{\nonTerminal{\sort{Exp}}}}\terminal{/}{{\nonTerminal{\sort{Exp}}}}}{strict}\syntaxCont{{{\nonTerminal{\sort{Exp}}}}\terminal{\%}{{\nonTerminal{\sort{Exp}}}}}{strict}\syntaxCont{{{\nonTerminal{\sort{Exp}}}}\terminal{+}{{\nonTerminal{\sort{Exp}}}}}{strict}\syntaxCont{{{\nonTerminal{\sort{Exp}}}}\terminal{-}{{\nonTerminal{\sort{Exp}}}}}{strict}\syntaxCont{{{\nonTerminal{\sort{Exp}}}}\terminal{<}{{\nonTerminal{\sort{Exp}}}}}{strict, non-assoc}\syntaxCont{{{\nonTerminal{\sort{Exp}}}}\terminal{<=}{{\nonTerminal{\sort{Exp}}}}}{strict, non-assoc}\syntaxCont{{{\nonTerminal{\sort{Exp}}}}\terminal{>}{{\nonTerminal{\sort{Exp}}}}}{strict, non-assoc}\syntaxCont{{{\nonTerminal{\sort{Exp}}}}\terminal{>=}{{\nonTerminal{\sort{Exp}}}}}{strict, non-assoc}\syntaxCont{{{\nonTerminal{\sort{Exp}}}}\terminal{==}{{\nonTerminal{\sort{Exp}}}}}{strict, non-assoc}\syntaxCont{{{\nonTerminal{\sort{Exp}}}}\terminal{!=}{{\nonTerminal{\sort{Exp}}}}}{strict, non-assoc}\syntaxCont{\terminal{not}{{\nonTerminal{\sort{Exp}}}}}{strict}\syntaxCont{{{\nonTerminal{\sort{Exp}}}}\terminal{and}{{\nonTerminal{\sort{Exp}}}}}{strict}\syntaxCont{{{\nonTerminal{\sort{Exp}}}}\terminal{or}{{\nonTerminal{\sort{Exp}}}}}{strict}\syntaxCont{{{\nonTerminal{\sort{Exp}}}}\terminal{=}{{\nonTerminal{\sort{Exp}}}}}{strict(2)}
\end{syntaxBlock}
\begin{kblock}[text]
 \subsection{Statements}
Most of the statement constructs are standard for imperative languages.
We syntactically distinguish between empty and non-empty blocks, because we
chose \textit{Stmts} not to be a (``\texttt{;}''-separated) list of
\textit{Stmt}.  Variables can be declared anywhere inside a block, their scope
ending with the block.  Expressions are allowed to be used for their side
effects only (followed by a semicolon ``\texttt{;}'').  Functions are allowed
to abruptly return.  The exceptions are parametric, i.e., one can throw a value
which is bound to the variable declared by \texttt{catch}.  Threads can be
dynamically created and terminated, and can synchronize with \texttt{acquire},
\texttt{release} and \texttt{rendezvous}.  Note that the strictness attributes
obey the intended evaluation strategy of the various constructs.  In
particular, the if-then-else construct is strict only in its first argument
(the if-then construct will be desugared into if-then-else), while the loops
constructs are not strict in any arguments.  The \texttt{print} statement
constructs is variadic, that is, it takes an arbitrary number of arguments. \end{kblock}

\begin{syntaxBlock}{\nonTerminal{\sort{Stmt}}}\syntax{\{\}}{}\syntaxCont{\{{{\nonTerminal{\sort{Stmts}}}}\}}{}\syntaxCont{{{\nonTerminal{\sort{Exp}}}}\terminal{;}}{strict}\syntaxCont{\terminal{if}{{\nonTerminal{\sort{Exp}}}}\terminal{then}{{\nonTerminal{\sort{Stmt}}}}\terminal{else}{{\nonTerminal{\sort{Stmt}}}}}{avoid, strict(1)}\syntaxCont{\terminal{if}{{\nonTerminal{\sort{Exp}}}}\terminal{then}{{\nonTerminal{\sort{Stmt}}}}}{}\syntaxCont{\terminal{while}{{\nonTerminal{\sort{Exp}}}}\terminal{do}{{\nonTerminal{\sort{Stmt}}}}}{}\syntaxCont{\terminal{for}{{\nonTerminal{\sort{Id}}}}\terminal{=}{{\nonTerminal{\sort{Exp}}}}\terminal{to}{{\nonTerminal{\sort{Exp}}}}\terminal{do}{{\nonTerminal{\sort{Stmt}}}}}{}\syntaxCont{\terminal{return}{{\nonTerminal{\sort{Exp}}}}\terminal{;}}{strict}\syntaxCont{\terminal{return}\terminal{;}}{}\syntaxCont{\terminal{write}({{\nonTerminal{\sort{Exp}}}})\terminal{;}}{strict}\syntaxCont{\terminal{try}{{\nonTerminal{\sort{Stmt}}}}\terminal{catch}({{\nonTerminal{\sort{Id}}}}){{\nonTerminal{\sort{Stmt}}}}}{}\syntaxCont{\terminal{throw}{{\nonTerminal{\sort{Exp}}}}\terminal{;}}{strict}\syntaxCont{\terminal{spawn}{{\nonTerminal{\sort{Stmt}}}}}{}\syntaxCont{\terminal{acquire}{{\nonTerminal{\sort{Exp}}}}\terminal{;}}{strict}\syntaxCont{\terminal{release}{{\nonTerminal{\sort{Exp}}}}\terminal{;}}{strict}\syntaxCont{\terminal{rendezvous}{{\nonTerminal{\sort{Exp}}}}\terminal{;}}{strict}\syntaxCont{\terminal{print}({{\nonTerminal{\sort{Exps}}}})\terminal{;}}{strict}
\end{syntaxBlock}

\begin{syntaxBlock}{\nonTerminal{\sort{Stmts}}}\syntax{{\nonTerminal{\sort{Decl}}}}{}\syntaxCont{{\nonTerminal{\sort{Stmt}}}}{}\syntaxCont{{{\nonTerminal{\sort{Stmts}}}}\mathrel{}{{\nonTerminal{\sort{Stmts}}}}}{}
\end{syntaxBlock}
\begin{kblock}[text]
 \section{Desugared Syntax}
This part desugars some of SIMPLE's language constructs into core ones.
We only want to give semantics to core constructs, so we get rid of the
derived ones before we start the semantics.  All desugaring macros below are
straightforward.  For the semantics, we can therefore assume that all
functions take a list of arguments, that each conditional has both branches,
that there are only \texttt{while} loops, and that each variable is
declared alone and is initialized. \end{kblock}
\krule{
\reduce{\terminal{if}{\variable[Exp]{E}}\terminal{then}{\variable[Stmt]{S}}}{\terminal{if}{\variable[Exp]{E}}\terminal{then}{\variable[Stmt]{S}}\terminal{else}{\{\}}}}{}{}
\krule{
\reduce{\terminal{for}{\variable[Id]{X}}\terminal{=}{\variable[K]{E1}}\terminal{to}{\variable[K]{E2}}\terminal{do}{\variable[K]{S}}}{\{{{\terminal{var}{{\variable[Id]{X}}\terminal{=}{\variable[K]{E1}}}\terminal{;}}\mathrel{}{\terminal{while}{{\variable[Id]{X}}\terminal{<=}{\variable[K]{E2}}}\terminal{do}{\{{{\variable[K]{S}}\mathrel{}{{{\variable[Id]{X}}\terminal{=}{{\variable[Id]{X}}\terminal{+}{\constant[\#Int]{1}}}}\terminal{;}}}\}}}}\}}}{}{}
\krule{
\reduce{\terminal{var}{{\variable[Exp]{E1}}\mathpunct{\terminalNoSpace{,}}{\variable[Exps]{Es}}}\terminal{;}}{{\terminal{var}{\variable[Exp]{E1}}\terminal{;}}\mathrel{}{\terminal{var}{{\variable[Exp]{E2}}\mathpunct{\terminalNoSpace{,}}{\variable[Exps]{Es}}}\terminal{;}}}}{}{}
\krule{
\reduce{\terminal{var}{{\variable[Id]{X}}\terminal{=}{\variable[Exp]{E}}}\terminal{;}}{{\terminal{var}{\variable[Id]{X}}\terminal{;}}\mathrel{}{{{\variable[Id]{X}}\terminal{=}{\variable[Exp]{E}}}\terminal{;}}}}{}{}

\begin{syntaxBlock}{\nonTerminal{\sort{\#Id}}}\syntax{\terminal{main}}{}
\end{syntaxBlock}
\end{module}
\begin{module}{\moduleName{SIMPLE-UNTYPED}}
\begin{kblock}[text]
 \section{Basic Semantic Infrastructure}
Before one starts adding semantic rules to a \K definition, one needs to
define the basic semantic infrastructure consisting of definitions for
{\em values} and {\em configuration}.  As discussed in the \K
definition of IMP, the values are needed to know when to stop applying
the heating rules and when to start applying the cooling rules
corresponding to strictness or context declarations.  The
configuration serves as a backbone for the process of configuration
abstraction which allows users to only mention the relevant cells in each
semantic rule, the rest of the configuration context being inferred
automatically.  Although the configuration could potentially be automatically
inferred from the rules, we believe that it is very useful for language
designers/semanticists to actually think of and design their configuration
explicitly, so the current implementation of \K requires one to define it. \end{kblock}
\begin{kblock}[text]
 \subsection{Values}
We here define the values of the language that the various fragments of
programs evaluate to.  First, integers and Booleans are values.  As discussed,
arrays evaluate to special array reference values holding (1) a location from
where the array's elements are contiguously allocated in the store, and
(2) the size of the array.  Functions evaluate to function values as
$\lambda$-abstractions (we do not need to evaluate functions to closures
because each function is executed in the fixed global environment and
function definitions cannot be nested).  Like in IMP and other
languages, we finally tell the tool that values are \K results. \end{kblock}

\begin{syntaxBlock}{\nonTerminal{\sort{Val}}}\syntax{{\nonTerminal{\sort{Int}}}}{}\syntaxCont{{\nonTerminal{\sort{Bool}}}}{}\syntaxCont{{\nonTerminal{\sort{String}}}}{}\syntaxCont{\terminal{arrayRef}({{\nonTerminal{\sort{Int}}}},{{\nonTerminal{\sort{Int}}}})}{}\syntaxCont{\terminal{lambda}({{\nonTerminal{\sort{Ids}}}},{{\nonTerminal{\sort{Stmt}}}})}{}
\end{syntaxBlock}

\begin{syntaxBlock}{\nonTerminal{\sort{Vals}}}\syntax{List\{Val, \mbox{``},\mbox{''}\}}{}
\end{syntaxBlock}

\begin{syntaxBlock}{\nonTerminal{\sort{Exp}}}\syntax{{\nonTerminal{\sort{Val}}}}{}
\end{syntaxBlock}

\begin{syntaxBlock}{\nonTerminal{\sort{KResult}}}\syntax{{\nonTerminal{\sort{Val}}}}{}
\end{syntaxBlock}
\begin{kblock}[text]
 The inclusion of values in expressions follows the methodology of
syntactic definitions (like, e.g., in SOS): extend the syntax of the language
to encompass all values and additional constructs needed to give semantics.
In addition to that, it allows us to write the semantic rules using the
original syntax of the language, and to parse them with the same (now extended
with additional values) parser.  If writing the semantics directly on the \K
AST, using the associated labels instead of the syntactic constructs, then one
would not need to include values in expressions. \end{kblock}
\begin{kblock}[text]
 \section{Configuration}
The \K configuration of SIMPLE consists of a top level cell, \textsf{T},
holding a \textsf{threads} cell, a global environment map cell \textsf{genv}
mapping the global variables and function names to their locations, a shared
store map cell \textsf{store} mapping each location to some value, a set cell
\textsf{busy} holding the locks which have been acquired but not yet released
by threads, \textsf{input} and \textsf{output} list cells, and a
\textsf{nextLoc} cell holding a natural number indicating the next available
location.  For simplicity, the location counter in \textsf{nextLoc} models an
actual physical location in the store (and assumes no garbage collection).
In other definitions (such as KERNELC) we show how one can model locations in
the store to be symbolic and thus abstract away form the memory allocator
library.   The \textsf{threads} cell contains one \textsf{thread} cell for
each existing thread in the program.  Note that the thread cell has
multiplicity ``*'', which means that at any given moment there could be zero,
one or more \textsf{thread} cells.  Each \textsf{thread} cell contains a
computation cell \textsf{k}, a \textsf{control} cell holding the various
control structures needed to jump to certain points of interest in the program
execution, a local environment map cell \textsf{env} mapping the thread local
variables to locations in the store, and finally a \textsf{holds} map cell
indicating what locks have been acquired by the thread and not released so far
and how many times (SIMPLE's locks are re-entrant).  The \textsf{control} cell
currently contains only two subcells, a function stack \textsf{fstack} which
is a list and an exception stack \textsf{xstack} which is also a list.
One can add more control structures in the \textsf{control} cell, such as a
stack for break/continue of loops, etc., if the language is extended with more
control-changing constructs.  Note that all cells except for \textsf{k} are
also initialized, in that they contain a ground term of their corresponding
sort.  The \textsf{k} cell is initialized with the program that will be passed
to the \K tool, as indicated by the \textit{\$PGM} variable. \end{kblock}
\kconfig{\kall[red]{T}{\kall[orange]{threads}{\kall[yellow]{thread}{\kall[green]{k}{\variable[K]{\$PGM}\kra\terminal{execute}}
\mathrel{}\kall[cyan]{control}{\kall[blue]{fstack}{\dotCt{List}}
\mathrel{}\kall[purple]{xstack}{\dotCt{List}}
}
\mathrel{}\kall[violet]{env}{\dotCt{Map}}
\mathrel{}\kall[black]{holds}{\dotCt{Map}}
}
}
\mathrel{}\kall[pink]{genv}{\dotCt{Map}}
\mathrel{}\kall[white]{store}{\dotCt{Map}}
\mathrel{}\kall[cyan]{busy}{\dotCt{Set}}
\mathrel{}\kall[magenta]{in}{\dotCt{List}}
\mathrel{}\kall[brown]{out}{\dotCt{List}}
\mathrel{}\kall[gray]{nextLoc}{\constant[\#Int]{0}}
}
}
\begin{kblock}[text]
 \section{Declarations and Initialization}
We start by defining the semantics of declarations (for variables,
arrays and functions). \end{kblock}
\begin{kblock}[text]
 \subsection{Variable Declaration}
The SIMPLE syntax was desugared above so that each variable is
declared alone and its initialization is done as a separate statement.
The semantic rule below matches resulting variable declarations of the
form ``$\texttt{var}\,X\texttt{;}$'' on top of the \textsf{k} cell
(indeed, note that the \textsf{k} cell is complete, or round, to the
left, and is torn, or ruptured, to the right), allocates a fresh
location $L$ in the store which is initialized with a special value
$\bot$ (indeed, the unit ``$\kdot$'', or nothing, is matched anywhere
in the map---note the tears at both sides---and replaced with the
mapping $L\mapsto \bot$), and binds $X$ to $L$ in the local
environment shadowing previous declarations of $X$, if any.  It is
this possible shadowing of $X$ which disallows us to use a similar
technique for updating the environment as for updating the store, as
we know that $L$ is not already bound in the store when we add $L
\mapsto \bot$.  We prefer the approach used for updating the store
whenever possible, because it offers more true concurrency than the
latter; indeed, according to the concurrent semantics of $K$, the
store is not frozen while $L\mapsto \bot$ is added to it, while the
environment is frozen during the update operation
$\textit{Env}[L/X]$.  The variable declaration command is also removed
from the top of the computation cell and the fresh location counter is
incremented.  All the above happen in one transactional step, with the
rule below.  Note also how configuration abstraction allows us to only
mention the needed cells; indeed, as the configuration above states,
the \textsf{k} and \textsf{env} cells are actually located within a
\textsf{thread} cell within the \textsf{threads} cell, but one needs
not mention these: the configuration context of the rule is
automatically transformed to match the declared configuration
structure.

\paragraph{Note:}{The "trick" with using a \textsf{nextLoc} cell to generate
fresh locations is rather low level and hopefully temporary; we intend to
soon allow instead a side-condition of the form ``where $L$ fresh''.} \end{kblock}

\begin{syntaxBlock}{\nonTerminal{\sort{K}}}\syntax{\terminal{undefined}}{}
\end{syntaxBlock}
\krule{
\kprefix[green]{k}{\reduce{\terminal{var}{\variable[Id]{X}}\terminal{;}}{\dotCt{K}}}
\mathrel{}\kall[violet]{env}{\reduce{\variable[Map]{Env}}{{\variable[Map]{Env}}[{\variable[Int]{L}}\terminal{/}{\variable[Id]{X}}]}}
\mathrel{}\kmiddle[white]{store}{\reduce{\dotCt{Map}}{\variable[Int]{L}\mapsto\terminal{undefined}}}
\mathrel{}\kall[gray]{nextLoc}{\reduce{\variable[Int]{L}}{{\variable[Int]{L}}\mathrel{+_{\scriptstyle\it Int}}{\constant[\#Int]{1}}}}
}{}{}
\begin{kblock}[text]
 \subsection{Array Declaration}
The \K semantics of the uni-dimensional array declaration is somehow similar
to the above declaration of ordinary variables.  $N+1$ locations are allocated
in the store for an array of size $N$, the additional location (chosen to be
the first one allocated) holding the array reference value.  The array
reference value \texttt{arrayRef(L,N)} states that the array has size
$N$ and its elements are located contiguously in the store starting
with location $L$.  Recall that $L..L'$ is the list of locations
between $L$ and $L'$ and that $L..L'\mapsto V$ initializes each
location in the list $L..L'$ to $V$.  Note that, since the dimensions
of array declarations can be arbitrary expressions, this virtually
means that we can dynamically allocate memory in SIMPLE by means of
array declarations. \end{kblock}
\krule{
\kprefix[green]{k}{\reduce{\terminal{var}{{\variable[Id]{X}}[{\variable[Int]{N}}]}\terminal{;}}{\dotCt{K}}}
\mathrel{}\kall[violet]{env}{\reduce{\variable[Map]{Env}}{{\variable[Map]{Env}}[{\variable[Int]{L}}\terminal{/}{\variable[Id]{X}}]}}
\mathrel{}\kmiddle[white]{store}{\reduce{\dotCt{Map}}{\variable[Int]{L}\mapsto\terminal{arrayRef}({{\variable[Int]{L}}\mathrel{+_{\scriptstyle\it Int}}{\constant[\#Int]{1}}},{\variable[Int]{N}})\mathrel{}{{\variable[Int]{L}}\mathrel{+_{\scriptstyle\it Int}}{\constant[\#Int]{1}}}\terminal{..}{{\variable[Int]{L}}\mathrel{+_{\scriptstyle\it Int}}{\variable[Int]{N}}}\mapsto\terminal{undefined}}}
\mathrel{}\kall[gray]{nextLoc}{\reduce{\variable[Int]{L}}{{{\variable[Int]{L}}\mathrel{+_{\scriptstyle\it Int}}{\constant[\#Int]{1}}}\mathrel{+_{\scriptstyle\it Int}}{\variable[Int]{N}}}}
}{}{}
\begin{kblock}[text]
 SIMPLE allows multi-dimensional arrays.  For semantic simplicity, we
desugar them all into uni-dimensional arrays by code transformation.
This way, we will only need to give semantics to uni-dimensional arrays.
First, the context rule below is used to request the evaluation of array
dimensions: \end{kblock}
\kcontext{
\terminal{var}{{{\variable[Id]{X}}[{\khole{}}]}\mathpunct{\terminalNoSpace{,}}{\dotCt{Exps}}}\terminal{;}}{}{}
\begin{kblock}[text]
  Upon evaluating the array dimensions, the code generation rule below
desugars multi-dimensional array declaration to uni-dimensional declarations.
To this aim, we introduce two special unique variable identifiers,
\texttt{\$1} and \texttt{\$2}.  The first, \texttt{\$1}, is assigned the array
reference value of the current array, so that we can redeclare the array
inside the loop body with fewer dimensions.  The second variable,
\texttt{\$2}, iterates through and initializes each element of the current
dimension: \end{kblock}

\begin{syntaxBlock}{\nonTerminal{\sort{\#Id}}}\syntax{\terminal{\$1}}{}\syntaxCont{\terminal{\$2}}{}
\end{syntaxBlock}
\krule{
\reduce{\terminal{var}{{\variable[Id]{X}}[{{\variable[Int]{N1}}\mathpunct{\terminalNoSpace{,}}{{\variable[Int]{N2}}\mathpunct{\terminalNoSpace{,}}{\variable[Vals]{Vs}}}}]}\terminal{;}}{{\terminal{var}{{\variable[Id]{X}}[{\variable[Int]{N1}}]}\terminal{;}}\mathrel{}{\{{{\terminal{var}{{\constant[\#Id]{\$1}}\terminal{=}{\variable[Id]{X}}}\terminal{;}}\mathrel{}{\terminal{for}{\constant[\#Id]{\$2}}\terminal{=}{\constant[\#Int]{0}}\terminal{to}{{\variable[Int]{N1}}\terminal{-}{\constant[\#Int]{1}}}\terminal{do}{\{{{\terminal{var}{{\variable[Id]{X}}[{{\variable[Int]{N2}}\mathpunct{\terminalNoSpace{,}}{\variable[Vals]{Vs}}}]}\terminal{;}}\mathrel{}{{{{\constant[\#Id]{\$1}}[{\constant[\#Id]{\$2}}]}\terminal{=}{\variable[Id]{X}}}\terminal{;}}}\}}}}\}}}}{}{structural}
\begin{kblock}[text]
 Ideally, one would like to perform syntactic desugarings like the one
above before the actual semantics.  Unfortunately, that was not possible in
this case because the dimension expressions of the multi-dimensional array need
to be evaluated first.  Indeed, the desugaring rule above does not work if the
dimensions of the declared array are arbitrary expressions, because they can
have side effects (e.g., \texttt{a[++x,++x]}) and those side effects would be
propagated each time the expression is evaluated in the desugaring code (note
that both the loop condition and the nested multi-dimensional declaration
would need to evaluate the expressions given as array dimensions). \end{kblock}
\begin{kblock}[text]
 \subsection{Function declaration}
Functions are evaluated to $\lambda$-abstractions and stored like any other
values in the store.  A binding is added into the environment for the function
name to the location holding its body.  Similarly to the C language, SIMPLE
only allows function declarations at the top level of the program.  More
precisely, the subsequent semantics of SIMPLE only works well when one
respects this requirement.  Indeed, the simplistic context-free parser
generated by the grammar above is more generous than we may want, in that it
allows function declarations anywhere any declaration is allowed, including
inside arbitrary blocks.  However, as the rule below shows, we are {\em not}
storing the declaration environment with the $\lambda$-abstraction value as
closures do.  Instead, as seen shortly, we switch to the global environment
whenever functions are invoked, which is consistent with our requirement that
functions should only be declared at the top.  Thus, if one declares local
functions, then one may see unexpected behaviors (e.g., when one shadows a
global variable before declaring a local function).  The type checker of
SIMPLE, also defined in \K (see simple/typed/static),
discards programs which do not respect this requirement. \end{kblock}
\krule{
\kprefix[green]{k}{\reduce{\terminal{function}{\variable[Id]{F}}({\variable[Ids]{Xs}}){\variable[Stmt]{S}}}{\dotCt{K}}}
\mathrel{}\kall[violet]{env}{\reduce{\variable[Map]{Env}}{{\variable[Map]{Env}}[{\variable[Int]{L}}\terminal{/}{\variable[Id]{F}}]}}
\mathrel{}\kmiddle[white]{store}{\reduce{\dotCt{Map}}{\variable[Int]{L}\mapsto\terminal{lambda}({\variable[Ids]{Xs}},{\variable[Stmt]{S}})}}
\mathrel{}\kall[gray]{nextLoc}{\reduce{\variable[Int]{L}}{{\variable[Int]{L}}\mathrel{+_{\scriptstyle\it Int}}{\constant[\#Int]{1}}}}
}{}{}
\begin{kblock}[text]
 When we are done with the first pass (pre-processing), the computation
cell \textsf{k} contains only the token \texttt{execute} (see the module
SIMPLE-UNTYPED below, whose rule for initiating the execution of a program
adds the token \texttt{execute} at the end of the program) and the cell
\textsf{genv} is empty.  In this case, we have to call \texttt{main()} and to
initialize the global environment by transferring the contents of the local
environment into it.  We prefer to do it this way, as opposed to processing
all the top level declarations directly within the global environment, because
we want to avoid duplication of semantics: the syntax of the global
declarations is identical to that of their corresponding local declarations,
so the semantics of the latter suffices provided that we copy the local
environment into the global one once we are done with the pre-processing.
We want this separate pre-processing step precisely because we want to create
the global environment.  All (top-level) functions end up having their names
bound in the global environment and, as seen below, they are executed in that
same global environment; all these mean, in particular, that the functions
``see'' each other, allowing for mutual recursion, etc. \end{kblock}

\begin{syntaxBlock}{\nonTerminal{\sort{K}}}\syntax{\terminal{execute}}{}
\end{syntaxBlock}
\krule{
\kall[green]{k}{\reduce{\terminal{execute}}{{{\constant[\#Id]{main}}({\dotCt{Exps}})}\terminal{;}}}
\mathrel{}\kall[violet]{env}{\variable[Map]{Env}}
\mathrel{}\kall[pink]{genv}{\reduce{\dotCt{Map}}{\variable[Map]{Env}}}
}{}{structural}
\begin{kblock}[text]
 \section{Expressions}
We next define the \K semantics of all the expression constructs, in the
order in which their syntax was declared. \end{kblock}
\begin{kblock}[text]
 \subsection{Variable lookup}
When a variable $X$ is the first computational task (note the rupture of the
\textsf{k} cell to the right), and $X$ is bound to some location $L$ in the
environment (note the rupture of the \textsf{env} cell at both sides), and
$L$ is mapped to some value $V$ in the store, then rewrite $X$ by $V$: \end{kblock}
\krule[look-up]{
\kprefix[green]{k}{\reduce{\variable[Id]{X}}{\variable[Val]{V}}}
\mathrel{}\kmiddle[violet]{env}{\variable[Id]{X}\mapsto\variable[Int]{L}}
\mathrel{}\kmiddle[white]{store}{\variable[Int]{L}\mapsto\variable[Val]{V}}
}{}{transition}
\begin{kblock}[text]
 Note that this excludes reading $\bot$, as $\bot$ is not a value. \end{kblock}
\begin{kblock}[text]
 \subsection{Variable/Array increment}
This is tricky, because we want to allow both {\tt ++x} and {\tt ++a[5]}.
Therefore, we need to extract the lvalue of the expression to increment.
To do that, we state that the expression to increment should be wrapped
by the auxiliary ``lvalue'' operation and then evaluated.
The semantics of the auxiliary lvalue operation is defined below.
For now, all we need to know is that it takes an expression and evaluates
to a location value, also introduced below with the auxiliary operations. \end{kblock}
\kcontext{
\terminal{++}{\reduce{\khole{}}{\terminal{lvalue}({\khole{}})}}}{}{}
\krule{
\kprefix[green]{k}{\reduce{\terminal{++}{\terminal{loc}({\variable[Int]{L}})}}{{\variable[Int]{I}}\mathrel{+_{\scriptstyle\it Int}}{\constant[\#Int]{1}}}}
\mathrel{}\kmiddle[white]{store}{\variable[Int]{L}\mapsto\reduce{\variable[Int]{I}}{{\variable[Int]{I}}\mathrel{+_{\scriptstyle\it Int}}{\constant[\#Int]{1}}}}
}{}{transition}
\begin{kblock}[text]
 \subsection{Arithmetic operators}
There is nothing special about the following rules.  They rewrite the
language constructs to their library counterparts when their arguments
become values of expected sorts: \end{kblock}
\krule{
\reduce{{\variable[Int]{I1}}\terminal{+}{\variable[Int]{I2}}}{{\variable[Int]{I1}}\mathrel{+_{\scriptstyle\it Int}}{\variable[Int]{I2}}}}{}{}
\krule{
\reduce{{\variable[String]{Str1}}\terminal{+}{\variable[String]{Str2}}}{{\variable[String]{Str1}}+_{\scriptstyle\it String}{\variable[String]{Str2}}}}{}{}
\krule{
\reduce{{\variable[Int]{I1}}\terminal{-}{\variable[Int]{I2}}}{{\variable[Int]{I1}}\mathrel{-_{\scriptstyle\it Int}}{\variable[Int]{I2}}}}{}{}
\krule{
\reduce{{\variable[Int]{I1}}\terminal{*}{\variable[Int]{I2}}}{{\variable[Int]{I1}}\mathrel{\ast_{\scriptstyle\it Int}}{\variable[Int]{I2}}}}{}{}
\krule{
\reduce{{\variable[Int]{I1}}\terminal{/}{\variable[Int]{I2}}}{{\variable[Int]{I1}}\mathrel{\div_{\scriptstyle\it Int}}{\variable[Int]{I2}}}}{{\variable[Int]{I2}}\terminal{=/=K}{\constant[\#Int]{0}}}{}
\krule{
\reduce{{\variable[Int]{I1}}\terminal{\%}{\variable[Int]{I2}}}{{\variable[Int]{I1}}\mathrel{\%_{\scriptstyle\it Int}}{\variable[Int]{I2}}}}{{\variable[Int]{I2}}\terminal{=/=K}{\constant[\#Int]{0}}}{}
\krule{
\reduce{\terminal{-}{\variable[Int]{I}}}{{\constant[\#Int]{0}}\mathrel{-_{\scriptstyle\it Int}}{\variable[Int]{I}}}}{}{}
\krule{
\reduce{{\variable[Int]{I1}}\terminal{<}{\variable[Int]{I2}}}{{\variable[Int]{I1}}\mathrel{<_{\scriptstyle\it Int}}{\variable[Int]{I2}}}}{}{}
\krule{
\reduce{{\variable[Int]{I1}}\terminal{<=}{\variable[Int]{I2}}}{{\variable[Int]{I1}}\mathrel{\leq_{\scriptstyle\it Int}}{\variable[Int]{I2}}}}{}{}
\krule{
\reduce{{\variable[Int]{I1}}\terminal{>}{\variable[Int]{I2}}}{{\variable[Int]{I1}}\mathrel{>_{\scriptstyle\it Int}}{\variable[Int]{I2}}}}{}{}
\krule{
\reduce{{\variable[Int]{I1}}\terminal{>=}{\variable[Int]{I2}}}{{\variable[Int]{I1}}\mathrel{\geq_{\scriptstyle\it Int}}{\variable[Int]{I2}}}}{}{}
\krule{
\reduce{{\variable[Val]{V1}}\terminal{==}{\variable[Val]{V2}}}{{\variable[Val]{V1}}\terminal{==K}{\variable[Val]{V2}}}}{}{}
\krule{
\reduce{{\variable[Val]{V1}}\terminal{!=}{\variable[Val]{V2}}}{{\variable[Val]{V1}}\terminal{=/=K}{\variable[Val]{V2}}}}{}{}
\krule{
\reduce{{\variable[Bool]{B1}}\terminal{and}{\variable[Bool]{B2}}}{{\variable[Bool]{B1}}\wedge_{\scriptstyle\it Bool}{\variable[Bool]{B2}}}}{}{}
\krule{
\reduce{{\variable[Bool]{B1}}\terminal{or}{\variable[Bool]{B2}}}{{\variable[Bool]{B1}}\vee_{\scriptstyle\it Bool}{\variable[Bool]{B2}}}}{}{}
\krule{
\reduce{\terminal{not}{\variable[Bool]{B}}}{\neg_{\scriptstyle\it Bool}{\variable[Bool]{B}}}}{}{}
\begin{kblock}[text]
 \subsection{Array lookup}
Untyped SIMPLE does not check array bounds (the dynamically typed version of
it, in ../typed/dynamic, does check for array out of bounds).  The first rule
below desugars multi-dimensional array access to uni-dimensional array access;
recall that the array access operation was declared strict, so all
sub-expressions involved are already values at this stage.  The second rule
rewrites the array access to a lookup operation at a precise location; we
prefer to do it this way to avoid locking the store.  Recall that ``---'' is an
anonymous variable in \K matching any subterm (like in Prolog); informally,
``there is something there but we don't care what''.
The semantics of the \texttt{lookup} operation is straightforward. \end{kblock}
\krule{
\reduce{{\variable[Val]{V}}[{{\variable[Int]{N1}}\mathpunct{\terminalNoSpace{,}}{{\variable[Int]{N2}}\mathpunct{\terminalNoSpace{,}}{\variable[Vals]{Vs}}}}]}{{{\variable[Val]{V}}[{{\variable[Int]{N1}}\mathpunct{\terminalNoSpace{,}}{\dotCt{Exps}}}]}[{{\variable[Int]{N2}}\mathpunct{\terminalNoSpace{,}}{\variable[Vals]{Vs}}}]}}{}{structural, anywhere}
\krule{
\reduce{{\terminal{arrayRef}({\variable[Int]{L}},{\AnyVar[Int]})}[{\variable[Int]{N}}]}{\terminal{lookup}({{\variable[Int]{L}}\mathrel{+_{\scriptstyle\it Int}}{\variable[Int]{N}}})}}{}{structural, anywhere}

\begin{syntaxBlock}{\nonTerminal{\sort{K}}}\syntax{\terminal{lookup}({{\nonTerminal{\sort{Int}}}})}{}
\end{syntaxBlock}
\krule{
\kprefix[green]{k}{\reduce{\terminal{lookup}({\variable[Int]{L}})}{\variable[Val]{V}}}
\mathrel{}\kmiddle[white]{store}{\variable[Int]{L}\mapsto\variable[Val]{V}}
}{}{transition}
\begin{kblock}[text]
 \subsection{Size of an array}
The size of the array is stored in the array reference value, and the
\texttt{sizeOf} construct was declared strict, so: \end{kblock}
\krule{
\reduce{\terminal{sizeOf}({\terminal{arrayRef}({\AnyVar[Int]},{\variable[Int]{N}})})}{\variable[Int]{N}}}{}{}
\begin{kblock}[text]
 \subsection{Function call}
Function application was strict in both its arguments, so we can
assume that both the function and its arguments are evaluated to
values (the former expected to be a $\lambda$-abstraction).  The first
rule below matches a well-formed function application on top of the
computation and performs the following steps atomically: it switches
to the function body followed by ``\texttt{return;}'' (for the case in
which the function does not use an explicit return statement); it
pushes the remaining computation, the current environment, and the
current control data onto the function stack (the remaining
computation can thus also be discarded from the computation cell,
because an unavoidable subsequent \texttt{return} statement---see
above---will always recover it from the stack); it switches the
current environment (which is being pushed on the function stack) to
the global environment, which is where the free variables in the
function body should be looked up; it binds the formal parameters to
fresh locations in the new environment, and stores the actual
arguments to those locations in the store.  The second rule pops the
computation, the environment and the control data from the function
stack when a \texttt{return} statement is encountered as the next
computational task, passing the returned value to the popped
computation (the popped computation was the context in which the
returning function was called).  Note that the pushing/popping of the
control data is crucial.  Without it, one may have a function that
contains an exception block with a return statement inside, which
would put the \textsf{xstack} cell in an inconsistent state (since the
exception block modifies it, but that modification should be
irrelevant once the function returns).  We add an artificial
\texttt{nothing} value to the language, which is returned by the
nulary \texttt{return;} statements. \end{kblock}

\begin{syntaxBlock}{\nonTerminal{\sort{ListItem}}}\syntax{({{\nonTerminal{\sort{Map}}}},{{\nonTerminal{\sort{K}}}},{{\nonTerminal{\sort{Bag}}}})}{}
\end{syntaxBlock}
\krule{
\kall[green]{k}{\reduce{{\terminal{lambda}({\variable[Ids]{Xs}},{\variable[K]{S}})}({\variable[Vals]{Vs}})\kra\variable[K]{K}}{\terminal{bindto}({\variable[Ids]{Xs}},{\variable[Vals]{Vs}})\kra\variable[K]{S}\kra\terminal{return}\terminal{;}}}
\mathrel{}\kall[cyan]{control}{\kprefix[blue]{fstack}{\reduce{\dotCt{List}}{({\variable[Map]{Env}},{\variable[K]{K}},{\variable[Bag]{C}})}}
\mathrel{}\variable[Bag]{C}}
\mathrel{}\kall[violet]{env}{\reduce{\variable[Map]{Env}}{\variable[Map]{GEnv}}}
\mathrel{}\kall[pink]{genv}{\variable[Map]{GEnv}}
}{}{}
\krule{
\kall[green]{k}{\reduce{\terminal{return}{\variable[Val]{V}}\terminal{;}\kra\AnyVar[K]}{\variable[Val]{V}\kra\variable[K]{K}}}
\mathrel{}\kall[cyan]{control}{\kprefix[blue]{fstack}{\reduce{({\variable[Map]{Env}},{\variable[K]{K}},{\variable[Bag]{C}})}{\dotCt{List}}}
\mathrel{}\reduce{\AnyVar[Bag]}{\variable[Bag]{C}}}
\mathrel{}\kall[violet]{env}{\reduce{\AnyVar[Map]}{\variable[Map]{Env}}}
}{}{}

\begin{syntaxBlock}{\nonTerminal{\sort{Val}}}\syntax{\terminal{nothing}}{}
\end{syntaxBlock}
\krule{
\reduce{\terminal{return}\terminal{;}}{\terminal{return}{\terminal{nothing}}\terminal{;}}}{}{structural}
\begin{kblock}[text]
 The \texttt{bindto} auxilliary construct binds a list of variables
to a list of values.  Specifically, it allocates a fresh location for
each variable, binding the variable to that location in the
environment and writing the value to that location in the store. \end{kblock}

\begin{syntaxBlock}{\nonTerminal{\sort{K}}}\syntax{\terminal{bindto}({{\nonTerminal{\sort{Ids}}}},{{\nonTerminal{\sort{Vals}}}})}{}
\end{syntaxBlock}
\krule{
\kprefix[green]{k}{\terminal{bindto}({{\variable[Id]{X}}\mathpunct{\terminalNoSpace{,}}{\reduce{\variable[Ids]{Xs}}{\variable[Ids]{Xs}}}},{{\variable[Val]{V}}\mathpunct{\terminalNoSpace{,}}{\reduce{\variable[Vals]{Vs}}{\variable[Vals]{Vs}}}})}
\mathrel{}\kall[violet]{env}{\reduce{\variable[Map]{Env}}{{\variable[Map]{Env}}[{\variable[Int]{L}}\terminal{/}{\variable[Id]{X}}]}}
\mathrel{}\kmiddle[white]{store}{\reduce{\dotCt{Map}}{\variable[Int]{L}\mapsto\variable[Val]{V}}}
\mathrel{}\kall[gray]{nextLoc}{\reduce{\variable[Int]{L}}{{\variable[Int]{L}}\mathrel{+_{\scriptstyle\it Int}}{\constant[\#Int]{1}}}}
}{}{}
\krule{
\kprefix[green]{k}{\reduce{\terminal{bindto}({\dotCt{Ids}},{\dotCt{Vals}})}{\dotCt{K}}}
}{}{structural}
\begin{kblock}[text]
 \subsection{Read}
The \texttt{read()} expression construct simply evaluates to the next
input value, at the same time discarding the input value from the
\textsf{in} cell. \end{kblock}
\krule{
\kprefix[green]{k}{\reduce{\terminal{read}()}{\variable[Int]{I}}}
\mathrel{}\kprefix[magenta]{in}{\reduce{\variable[Int]{I}}{\dotCt{List}}}
}{}{transition}
\begin{kblock}[text]
 \subsection{Assignment}
In SIMPLE, like in C, assignments are expression constructs and not statement
constructs.  To make it a statement all one needs to do is to follow it by a
semi-colon ``\texttt{;}'' (see the semantics for expression statements below).
Like for the increment, we want to allow assignments not only to variables but
also to array elements, e.g., \texttt{e1[e2] = e3} where \texttt{e1} evaluates
to an array reference, \texttt{e2} to a natural number, and \texttt{e3} to any 
value.  Thus, we first compute the lvalue of the left-hand-side expression
that appears in an assignment.  Like for the increment, all we need to know is
that \texttt{lvalue(...)} eventually evaluates to a location value
\texttt{loc(...)}. \end{kblock}
\kcontext{
{\reduce{\khole{}}{\terminal{lvalue}({\khole{}})}}\terminal{=}{\AnyVar[K]}}{}{}
\krule{
\kprefix[green]{k}{\reduce{{\terminal{loc}({\variable[Int]{L}})}\terminal{=}{\variable[Val]{V}}}{\variable[Val]{V}}}
\mathrel{}\kmiddle[white]{store}{\variable[Int]{L}\mapsto\reduce{\AnyVar[List\{K\}]}{\variable[Val]{V}}}
}{}{transition}
\begin{kblock}[text]
 \section{Statements}
We next define the \K semantics of statements, also in the order their syntax
was given. \end{kblock}
\begin{kblock}[text]
 \subsection{Blocks}
Empty blocks are simply discarded, as shown in the first rule below.
For non-empty blocks, we schedule the enclosed statement but we have to
make sure the environment is recovered after the enclosed statement executes.
Recall that we allow local variable declarations, whose scope is the block
enclosing them.  That is the reason for which we have to recover the
environment after the block.  This allows us to have a very simple semantics
for variable declarations, as we did above.  One can make the two rules below
computational if one wants them to count as computational steps. \end{kblock}
\krule{
\reduce{\{\}}{\dotCt{K}}}{}{structural}
\krule{
\kprefix[green]{k}{\reduce{\{{\variable[Stmts]{Ss}}\}}{\variable[Stmts]{Ss}\kra\terminal{env}({\variable[Map]{Env}})}}
\mathrel{}\kall[violet]{env}{\variable[Map]{Env}}
}{}{structural}
\begin{kblock}[text]
 The basic definition of environment recovery is straightforward: \end{kblock}

\begin{syntaxBlock}{\nonTerminal{\sort{K}}}\syntax{\terminal{env}({{\nonTerminal{\sort{Map}}}})}{}
\end{syntaxBlock}
\krule{
\kprefix[green]{k}{\reduce{\terminal{env}({\variable[Map]{Env}})}{\dotCt{K}}}
\mathrel{}\kall[violet]{env}{\reduce{\AnyVar[Map]}{\variable[Map]{Env}}}
}{}{structural}
\begin{kblock}[text]
  While theoretically sufficient, the basic definition for environment
recovery alone is suboptimal.  Consider a loop \texttt{while E do S},
whose semantics (see below) is given by unrolling.  Typically \texttt{S}
is a block.  Then the semantics of blocks above, together with the
unrolling semantics of the while loop below, will yield a computation
structure in the \textsf{k} cell that increasingly grows, adding a new
environment recovery task right in front of the already existing sequence of
similar environment recovery tasks (this phenomenon is similar to the ``tail
recursion'' problem).  Of course, when we have a sequence of environment
recovery tasks, we only need to keep the last one.  The elegant rule below
does precisely that, thus avoiding the unnecessary computation explosion
problem:  \end{kblock}
\krule{
\reduce{\terminal{env}({\AnyVar[Map]})}{\dotCt{K}}\kra\terminal{env}({\AnyVar[Map]})}{}{structural}
\begin{kblock}[text]
 There are two common alternatives to the above semantics of blocks.
One is to keep track of the variables which are declared in the block and only
recover those at the end of the block.  This way one does more work for
variable declarations but conceptually less work for environment recovery; we
say ``conceptually'' because it is not clear that it is indeed the case that
one does less work when AC matching is involved.  The other alternative is to
work with a stack of environments instead of a flat environment, and push the
current environment when entering a block and pop it when exiting it.  This
way, one does more work when accessing variables (since one has to search the
variable in the environment stack in a top-down manner), but on the other hand
uses smaller environments and the definition gets closer to an implementation.
Based on experience with dozens of language semantics and other \K definitions,
we have found that our approach above is the best trade-off between elegance
and efficiency (especially since rewrite engines have built-in techniques to
lazily copy terms, by need, thus not creating unnecessary copies),
so it is the one that we follow in general. \end{kblock}
\begin{kblock}[text]
 \subsection{Sequential composition}
Sequential composition is desugared into \K's builtin sequentialization
operation (recall that, like in C, the semi-colon ``\texttt{;}'' is not a
statement separator in SIMPLE---it is either a statement terminator or a
construct for a statement from an expression).  The rule below is
structural, so it does not count as a computational step.  One can make it
computational if one wants it to count as a step.  Note that \K allows
to define the semantics of SIMPLE in such a way that statements eventually
dissolve from the top of the computation when they are completed; this is in
sharp contrast to (artificially) ``evaluating'' them to a special
\texttt{skip} statement value and then getting rid of that special value, as
it is the case in other semantic approaches (where everything must evaluate
to something).  This means that once $S_1$ completes in the rule below, $S_2$
becomes automatically the next computation item without any additional
(explicit or implicit) rules. \end{kblock}
\krule{
\reduce{{\variable[K]{S1}}\mathrel{}{\variable[K]{S2}}}{\variable[K]{S1}\kra\variable[K]{S2}}}{}{structural}
\begin{kblock}[text]
 \subsection{Expression statements}
Expression statements are only used for their side effects, so their result
value is simply discarded.  Common examples of expression statements are ones
of the form \texttt{++x;}, \texttt{x=e;}, \texttt{e1[e2]=e3;}, etc. \end{kblock}
\krule{
\reduce{{\variable[Val]{V}}\terminal{;}}{\dotCt{K}}}{}{}
\begin{kblock}[text]
 \subsection{Conditional}
Since the conditional was declared with the \texttt{strict(1)} attribute, we
can assume that its first argument will eventually be evaluated.  The rules
below cover the only two possibilities in which the conditional is allowed to
proceed (otherwise the rewriting process gets stuck). \end{kblock}
\krule{
\reduce{\terminal{if}{\constant[\#Bool]{true}}\terminal{then}{\variable[Stmt]{S}}\terminal{else}{\AnyVar[K]}}{\variable[Stmt]{S}}}{}{}
\krule{
\reduce{\terminal{if}{\constant[\#Bool]{false}}\terminal{then}{\AnyVar[K]}\terminal{else}{\variable[Stmt]{S}}}{\variable[Stmt]{S}}}{}{}
\begin{kblock}[text]
 \subsection{While loop}
The simplest way to give the semantics of the while loop is by unrolling.
Note, however, that its unrolling is only allowed when the while loop reaches
the top of the computation (to avoid non-termination of unrolling).  We prefer
the rule below to be structural, because we don't want the unrolling of the
while loop to count as a computational step; this is unavoidable in
conventional semantics, but it is possible in \K thanks to its distinction
between structural and computational rules.  The simple while loop semantics
below works because our while loops in SIMPLE are indeed very basic.  If we
allowed break/continue of loops then we would need a completely different
semantics, which would also involve the \textsf{control} cell. \end{kblock}
\krule{
\kprefix[green]{k}{\reduce{\terminal{while}{\variable[Exp]{E}}\terminal{do}{\variable[Stmt]{S}}}{\terminal{if}{\variable[Exp]{E}}\terminal{then}{\{{{\variable[Stmt]{S}}\mathrel{}{\terminal{while}{\variable[Exp]{E}}\terminal{do}{\variable[Stmt]{S}}}}\}}\terminal{else}{\{\}}}}
}{}{structural}
\begin{kblock}[text]
 \subsection{Print}
The \texttt{print} statement was strict, so all its arguments are now
evaluated (recall that \texttt{print} is variadic).  We append each of
its evaluated arguments to the output buffer, and discard the residual
\texttt{print} statement with an empty list of arguments. \end{kblock}
\krule{
\kprefix[green]{k}{\terminal{print}({{\variable[Val]{V}}\mathpunct{\terminalNoSpace{,}}{\reduce{\variable[Vals]{Vs}}{\variable[Vals]{Vs}}}})\terminal{;}}
\mathrel{}\ksuffix[brown]{out}{\reduce{\dotCt{List}}{\variable[Val]{V}}}
}{}{transition}
\krule{
\reduce{\terminal{print}({\dotCt{Vals}})\terminal{;}}{\dotCt{K}}}{}{structural}
\begin{kblock}[text]
 \subsection{Exceptions}
SIMPLE allows parametric exceptions, in that one can throw and catch a
particular value.  The statement ``\texttt{try $S_1$ catch($X$) $S_2$}''
proceeds with the evaluation of $S_1$.  If $S_1$ evaluates normally, i.e.,
without any exception thrown, then $S_2$ is discarded and the execution
continues normally.  If $S_1$ throws an exception with a statement of the
form ``\texttt{throw $E$}'', then $E$ is first evaluated to some value $V$
(\texttt{throw} was declared to be strict), then $V$ is bound to $X$, then
$S_2$ is evaluated in the new environment while the reminder of $S_1$ is
discarded, then the environment is recovered and the execution continues
normally with the statement following the ``\texttt{try $S_1$ catch($X$)
$S_2$}'' statement.  Exceptions can be nested and the statements in the
``\texttt{catch}'' part ($S_2$ in our case) can throw exceptions to the
upper level.  One should be careful with how one handles the control data
structures here, so that the abrupt changes of control due to exception
throwing and to function returns interact correctly with each other.
For example, we want to allow function calls inside the statement $S_1$ in
a ``\texttt{try $S_1$ catch($X$) $S_2$}'' block which can throw an exception
that is not caught by the function but instead is propagated to the
``\texttt{try $S_1$ catch($X$) $S_2$}'' block that called the function.
Therefore, we have to make sure that the function stack as well as other
potential control structures are also properly modified when the exception
is thrown to correctly recover the execution context.  This can be easily
achieved by pushing/popping the entire current control context onto the
exception stack.  The three rules below modularly do precisely the above. \end{kblock}

\begin{syntaxBlock}{\nonTerminal{\sort{ListItem}}}\syntax{({{\nonTerminal{\sort{Id}}}},{{\nonTerminal{\sort{Stmt}}}},{{\nonTerminal{\sort{K}}}},{{\nonTerminal{\sort{Map}}}},{{\nonTerminal{\sort{Bag}}}})}{}
\end{syntaxBlock}

\begin{syntaxBlock}{\nonTerminal{\sort{K}}}\syntax{\terminal{popx}}{}
\end{syntaxBlock}
\krule{
\kall[green]{k}{\reduce{\terminal{try}{\variable[Stmt]{S1}}\terminal{catch}({\variable[Id]{X}}){\variable[Stmt]{S2}}}{\variable[Stmt]{S1}\kra\terminal{popx}}\kra\variable[K]{K}}
\mathrel{}\kall[cyan]{control}{\kprefix[purple]{xstack}{\reduce{\dotCt{List}}{({\variable[Id]{X}},{\variable[Stmt]{S2}},{\variable[K]{K}},{\variable[Map]{Env}},{\variable[Bag]{C}})}}
\mathrel{}\variable[Bag]{C}}
\mathrel{}\kall[violet]{env}{\variable[Map]{Env}}
}{}{}
\krule{
\kprefix[green]{k}{\reduce{\terminal{popx}}{\dotCt{K}}}
\mathrel{}\kprefix[purple]{xstack}{\reduce{\AnyVar[ListItem]}{\dotCt{List}}}
}{}{}
\krule{
\kall[green]{k}{\reduce{\terminal{throw}{\variable[Val]{V}}\terminal{;}\kra\AnyVar[K]}{\{{{\terminal{var}{{\variable[Id]{X}}\terminal{=}{\variable[Val]{V}}}\terminal{;}}\mathrel{}{\variable[Stmt]{S2}}}\}\kra\variable[K]{K}}}
\mathrel{}\kall[cyan]{control}{\kprefix[purple]{xstack}{\reduce{({\variable[Id]{X}},{\variable[Stmt]{S2}},{\variable[K]{K}},{\variable[Map]{Env}},{\variable[Bag]{C}})}{\dotCt{List}}}
\mathrel{}\reduce{\AnyVar[Bag]}{\variable[Bag]{C}}}
\mathrel{}\kall[violet]{env}{\reduce{\AnyVar[Map]}{\variable[Map]{Env}}}
}{}{}
\begin{kblock}[text]
 The catch statement $S_2$ needs to be executed in the original environment,
but where the thrown value $V$ is bound to the catch variable $X$.  We here
chose to rely on two previously defined constructs when giving semantics to
the catch part of the statement: (1) the variable declaration with
initialization, for binding $X$ for $V$; and (2) the block construct for
preventing $X$ from shadowing variables in the original environment upon the
completion of $S_2$.  Note, however, that the semantics of {\tt throw} can
also be given directly, in one computational step, especially in languages
without variable initializers and blocks. \end{kblock}
\begin{kblock}[text]
 \subsection{Threads}
SIMPLE's threads can be created and terminated dynamically, and can
synchronize by acquiring and releasing re-entrant locks and by rendezvous.
We discuss the seven rules giving the semantics of these operations below. \end{kblock}
\begin{kblock}[text]
 \subsubsection{Thread creation}
Threads can be created by any other threads using the ``\texttt{spawn $S$}''
statement.  The spawn statement is consumed in the creating thread and, at the
same time, a new thread cell is added to the top of the configuration,
initialized with the $S$ statement and sharing the same environment with the
spawning thread.  Note that the newly created \textsf{thread} cell is torn.
That means that the remaining cells are added and initialized automatically as
described in the definition of SIMPLE's configuration.  This is part of \K's
configuration abstraction/concretization mechanism. \end{kblock}
\krule{
\kmiddle[yellow]{thread}{\kprefix[green]{k}{\reduce{\terminal{spawn}{\variable[Stmt]{S}}}{\dotCt{K}}}
\mathrel{}\kall[violet]{env}{\variable[Map]{Env}}
}
\mathrel{}\reduce{\dotCt{Bag}}{\kmiddle[yellow]{thread}{\kall[green]{k}{\variable[Stmt]{S}}
\mathrel{}\kall[violet]{env}{\variable[Map]{Env}}
}
}}{}{}
\begin{kblock}[text]
 \subsubsection{Thread termination}
Dually to the above, when a thread terminates its assigned computation (the
contents of its \textsf{k} cell) is empty, so the thread can be dissolved.
However, since no discipline is imposed on how locks are acquired and released,
it can be the case that a terminating thread still holds locks.  Those locks
must be released, so other threads attempting to acquire them do not deadlock.
We achieve that by removing all the locks held by the terminating thread in its
\textsf{holds} cell from the set of busy locks in the \textsf{busy} cell
(\texttt{keys($H$)} returns the domain of the map $H$ as a set, that is, only
the locks themselves ignoring their multiplicity).  As seen below, a lock is
added to the \textsf{busy} cell as soon as it is acquired for the first time
by a thread. \end{kblock}
\krule{
\reduce{\kmiddle[yellow]{thread}{\kall[green]{k}{\dotCt{K}}
\mathrel{}\kall[black]{holds}{\variable[Map]{H}}
}
}{\dotCt{Bag}}\mathrel{}\kall[cyan]{busy}{\reduce{\variable[Set]{Busy}}{{\variable[Set]{Busy}}\terminal{-Set}{\terminal{keys}{\variable[Map]{H}}}}}
}{}{}
\begin{kblock}[text]
 \subsubsection{Acquire lock}
There are two cases to distinguish when a thread attempts to acquire a lock
(in SIMPLE any value can be used as a lock):\\
(1) The thread does not currently have the lock, in which case it has to
take it provided that the lock is not already taken by another thread (see
the side condition of the first rule).\\
(2) The thread already has the lock, in which case it just increments its
counter for the lock (the locks are re-entrant).  These two cases are captured
by the two rules below: \end{kblock}
\krule{
\kprefix[green]{k}{\reduce{\terminal{acquire}{\variable[K]{V}}\terminal{;}}{\dotCt{K}}}
\mathrel{}\kmiddle[black]{holds}{\reduce{\dotCt{Map}}{\variable[K]{V}\mapsto\constant[\#Int]{0}}}
\mathrel{}\kall[cyan]{busy}{\variable[Set]{Busy}\mathrel{}\reduce{\dotCt{Set}}{\variable[K]{V}}}
}{\neg_{\scriptstyle\it Bool}{{\variable[K]{V}}\terminal{in}{\variable[Set]{Busy}}}}{transition}
\krule{
\kprefix[green]{k}{\reduce{\terminal{acquire}{\variable[Val]{V}}\terminal{;}}{\dotCt{K}}}
\mathrel{}\kmiddle[black]{holds}{\variable[Val]{V}\mapsto\reduce{\variable[Int]{N}}{{\variable[Int]{N}}\mathrel{+_{\scriptstyle\it Int}}{\constant[\#Int]{1}}}}
}{}{}
\begin{kblock}[text]
 \subsubsection{Release lock}
Similarly, there are two corresponding cases to distinguish when a thread
releases a lock:\\
(1) The thread holds the lock more than once, in which case all it needs to do
is to decrement the lock counter.\\
(2) The thread holds the lock only once, in which case it needs to remove it
from its \textsf{holds} cell and also from the the shared \textsf{busy} cell,
so other threads can acquire it if they need to. \end{kblock}
\krule{
\kprefix[green]{k}{\reduce{\terminal{release}{\variable[Val]{V}}\terminal{;}}{\dotCt{K}}}
\mathrel{}\kmiddle[black]{holds}{\variable[Val]{V}\mapsto\reduce{\variable[Int]{N}}{{\variable[Int]{N}}\mathrel{-_{\scriptstyle\it Int}}{\constant[\#Int]{1}}}}
}{{\variable[Int]{N}}\mathrel{>_{\scriptstyle\it Int}}{\constant[\#Int]{0}}}{}
\krule{
\kprefix[green]{k}{\reduce{\terminal{release}{\variable[Val]{V}}\terminal{;}}{\dotCt{K}}}
\mathrel{}\kmiddle[black]{holds}{\reduce{\variable[Val]{V}\mapsto\constant[\#Int]{0}}{\dotCt{Map}}}
\mathrel{}\kmiddle[cyan]{busy}{\reduce{\variable[Val]{V}}{\dotCt{Set}}}
}{}{}
\begin{kblock}[text]
 \subsubsection{Rendezvous synchronization}
In addition to synchronization through acquire and release of locks, SIMPLE
also provides a construct for rendezvous synchronization.  A thread whose next
statement to execute is \texttt{rendezvous($V$)} gets stuck until another
thread reaches an identical statement; when that happens, the two threads
drop their rendezvous statements and continue their executions.  If three
threads happen to have an identical rendezvous statement as their next
statement, then precisely two of them will synchronize and the other will
remain blocked until another thread reaches a similar rendezvous statement.
The rule below is as simple as it can be.  Note, however, that, again, it is
\K's mechanism for configuration abstraction that makes it work as desired:
since the only cell which can multiply containing a \textsf{k} cell inside is
the \textsf{thread} cell, the only way to concretize the rule below to the
actual configuration of SIMPLE is to include each \textsf{k} cell in a
\textsf{thread} cell. \end{kblock}
\krule{
\kprefix[green]{k}{\reduce{\terminal{rendezvous}{\variable[Val]{V}}\terminal{;}}{\dotCt{K}}}
\mathrel{}\kprefix[green]{k}{\reduce{\terminal{rendezvous}{\variable[Val]{V}}\terminal{;}}{\dotCt{K}}}
}{}{transition}
\begin{kblock}[text]
 \section{Auxiliary declarations and operations}\end{kblock}
\begin{kblock}[text]
 \subsection{lvalue and loc}
For convenience in giving the semantics of constructs like the increment and
the assignment, that we want to operate the same way on variables and on
array elements, we used an auxiliary \texttt{lvalue($E$)} construct which was
expected to evaluate to the lvalue of the expression $E$.  This is only
defined when $E$ has an lvalue, that is, when $E$ is either a variable or
evaluates to an array element.  \texttt{lvalue($E$)} evaluates to a value of
the form \texttt{loc($L$)}, where $L$ is the location where the value of $E$
can be found; for clarity, we use \texttt{loc} to structurally distinguish
natural numbers from location values.  In giving semantics to \texttt{lvalue}
there are two cases to consider.  (1) If $E$ is a variable, then all we need
to do is to grab its location from the environment.  (2) If $E$ is an array
element, then we first evaluate the array and its index in order to identify
the exact location of the element of concern, and then return that location;
the last rule below works because its preceding context declarations ensure
that the array and its index are evaluated, and then the rule for array lookup
(defined above) rewrites the evaluated array access construct to its
corresponding store lookup operation. \end{kblock}

\begin{syntaxBlock}{\nonTerminal{\sort{Exp}}}\syntax{\terminal{lvalue}({{\nonTerminal{\sort{K}}}})}{}
\end{syntaxBlock}

\begin{syntaxBlock}{\nonTerminal{\sort{Val}}}\syntax{\terminal{loc}({{\nonTerminal{\sort{Int}}}})}{}
\end{syntaxBlock}
\krule{
\kprefix[green]{k}{\reduce{\terminal{lvalue}({\variable[Id]{X}})}{\terminal{loc}({\variable[Int]{L}})}}
\mathrel{}\kmiddle[violet]{env}{\variable[Id]{X}\mapsto\variable[Int]{L}}
}{}{structural}
\kcontext{
\terminal{lvalue}({{\AnyVar[K]}[{\khole{}}]})}{}{}
\kcontext{
\terminal{lvalue}({{\khole{}}[{\AnyVar[K]}]})}{}{}
\krule{
\kprefix[green]{k}{\reduce{\terminal{lvalue}({\terminal{lookup}({\variable[Int]{L}})})}{\terminal{loc}({\variable[Int]{L}})}}
}{}{structural}
\begin{kblock}[text]
 \subsection{Sequences of locations}
The following operation expands to the list of natural numbers between two
given numbers.  The first number is expected to be no larger than the second.
The two rules below are structural, for the same reason as above. \end{kblock}

\begin{syntaxBlock}{\nonTerminal{\sort{List\{K\}}}}\syntax{{{\nonTerminal{\sort{Int}}}}\terminal{..}{{\nonTerminal{\sort{Int}}}}}{}
\end{syntaxBlock}
\krule{
\reduce{{\variable[Int]{N1}}\terminal{..}{\variable[Int]{N2}}}{\dotCt{List{K}}}}{{\variable[Int]{N1}}\mathrel{>_{\scriptstyle\it Int}}{\variable[Int]{N2}}}{structural, anywhere}
\krule{
\reduce{{\variable[Int]{N1}}\terminal{..}{\variable[Int]{N2}}}{\variable[Int]{N1}\kcomma{{\variable[Int]{N1}}\mathrel{+_{\scriptstyle\it Int}}{\constant[\#Int]{1}}}\terminal{..}{\variable[Int]{N2}}}}{{\variable[Int]{N1}}\mathrel{\leq_{\scriptstyle\it Int}}{\variable[Int]{N2}}}{structural, anywhere}
\begin{kblock}[text]
 The semantics of SIMPLE is now complete. \end{kblock}
\end{module}
\end{kdefinition}
\end{document}
