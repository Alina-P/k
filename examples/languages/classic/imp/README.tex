% Type the command ``make pdf'' to generate the PDF poster of this language.
% Alternatively, type in the equivalent command
% kompile imp.k -pdf IMP-SYNTAX IMP-SEMANTICS IMP -topmatter README.tex

\setlength{\parindent}{1em}
\title{IMP}
\author{Grigore Ro\c{s}u (\texttt{grosu@illinois.edu})}
\organization{University of Illinois at Urbana-Champaign}

\maketitle

\begin{latexComment}
\section{Abstract}
This is the \K semantic definition of the classic IMP language.
IMP is considered a folklore language, without an official inventor,
and has been used in many textbooks and papers, often with slight
syntactic variations and often without being called IMP.  It includes
the most basic imperative language constructs, namely basic constructs
for arithmetic and Boolean expressions, and variable assignment,
conditional, while loop and sequential composition constructs for statements.
\end{latexComment}

\vspace*{3ex}
