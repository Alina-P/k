\setlength{\parindent}{1em}
\title{IMP}
\author{Grigore Ro\c{s}u (\texttt{grosu@illinois.edu})}
\organization{University of Illinois at Urbana-Champaign}

\maketitle

\begin{latexComment}
\section{Abstract}
This is the \K semantic definition of the classic IMP language.
IMP is considered a folklore language, without an official inventor,
and has been used in many textbooks and papers, often with slight
syntactic variations and often without being called IMP.  It includes
the most basic imperative language constructs, namely basic constructs
for arithmetic and Boolean expressions, and variable assignment,
conditional, while loop and sequential composition constructs for statements.

\paragraph{Note:}{
Since IMP is so simple, we will use its semantics as an oportunity to
explain some of the main features of the \K  prototype tool.  Other
features will be explained in other \K definitions comming as part of
this distribution.  For a quick introduction to the \K prototype, you
are referred to the README file at the root of this k-framework
distribution.  If you are interested in reading more about \K, please
check the following paper:
\begin{quote}
Grigore Ro\c su, Traian-Florin \c Serb\u anu\c t\u a:
\href{http://dx.doi.org/10.1016/j.jlap.2010.03.012}{An overview of the K semantic framework}.

Journal of Logic and Algebraic Programming, 79(6): 397-434 (2010)
\end{quote}
}

\end{latexComment}

\vspace*{3ex}
