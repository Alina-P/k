\setlength{\parindent}{1em}
\title{IMP}
\author{Grigore Ro\c{s}u (\texttt{grosu@illinois.edu})}
\organization{University of Illinois at Urbana-Champaign}

\maketitle

\begin{latexComment}
\section{Abstract}
This is the \K semantic definition of the classic IMP language.
IMP is considered a folklore language, without an official inventor,
and has been used in many textbooks and papers, often with slight
syntactic variations and often without being called IMP\@.  It includes
the most basic imperative language constructs, namely basic constructs
for arithmetic and Boolean expressions, and variable assignment,
conditional, while loop and sequential composition constructs for statements.

\paragraph{Note:}{
Since IMP is so simple, we will use its semantics as an opportunity to
explain some of the main features of the \K  prototype tool.  Other
features will be explained in other \K definitions coming as part of
this distribution.  For a quick introduction to the \K prototype, you
are referred to the README file at the root of this k-framework
distribution.  If you are interested in reading more about \K, please
check the following paper:
\begin{quote}
Grigore Ro\c su, Traian-Florin \c Serb\u anu\c t\u a:
\href{http://dx.doi.org/10.1016/j.jlap.2010.03.012}
     {An overview of the K semantic framework}.

Journal of Logic and Algebraic Programming, 79(6): 397-434 (2010)
\end{quote}
}

\paragraph{Note:}{
\K follows the
\href{https://en.wikipedia.org/wiki/Literate_programming}{literate
programming} approach.  The various semantic features defined in a \K
module can be reordered at will and can be commented using normal
comments like in C/C++/Java.  If those comments start with
`\texttt{@}' preceded by no space (e.g.,
`\texttt{//@ {\textbackslash}section\{Variable declarations\}}')
then they are interpreted as formal Latex documentation by the
\texttt{kompile} tool when used with the option \texttt{--pdf}
(or \texttt{--latex}).
While comments are useful in general, they can annoy the expert user
of \K.  To turn them off, you can do one of the following (unless you
want to remove them manually): (1) Use an editor which can hide or
color conventional C-like comments; or (2) Run the \K pre-processor
(kpp), which removes all comments.
}

\paragraph{Note:}{
The \K tool provides modules for grouping language features.  While
there are no rigid requirements or even guidelines on how to group
features, we often put the language syntax in a separate module, so we
can experiment with parsing programs before we attempt to give our
language a semantics.  In the case of IMP, we call its syntax module
\texttt{IMP-SYNTAX} and its semantics module \texttt{IMP}.  To compile
the IMP definition into Maude, pass the \texttt{imp.k}
(extension is optional) file to \texttt{kompile}:
\begin{verbatim}
$ kompile imp.k
\end{verbatim}
To compile it to a PDF document for visualization, use the command:
\begin{verbatim}
$ kompile imp.k --pdf
\end{verbatim}
To run programs, simply pass the program to \texttt{krun} in the
directory where \texttt{imp.k} is:
\begin{verbatim}
$ krun programs/sumPgm.imp
\end{verbatim}
In general, the above commands work as shown for any language, say
LANG, provided that you replace \texttt{IMP} by \texttt{LANG} and
\texttt{imp} by \texttt{lang}, and provided that you type these
commands in the directory where your language definition
\texttt{lang.k} is.  If you want to use different naming
conventions or to call these commands from other places, then type
``\texttt{kompile -\,\!-help}'' and ``\texttt{krun -\,\!-help}'' for
instructions.
}

\end{latexComment}

\vspace*{3ex}
